% LLNCS macro package for Springer Computer Science proceedings;
% Version 2.20 of 2017/10/04
%
\documentclass[runningheads]{llncs}
%
\usepackage{graphicx}
\usepackage{lipsum}
\usepackage{amsmath}
\usepackage{authblk}
\usepackage{makecell,array}
%\usepackage{natbib}
%\usepackage[square,sort,comma,numbers]{natbib}
%\usepackage[numbers]{natbib}

\usepackage[square,numbers]{natbib}
\usepackage{graphicx}
\usepackage{float}
\usepackage{gensymb}
\usepackage{wrapfig}
\usepackage{amssymb}
\usepackage{url}
\usepackage{mathtools}
\usepackage{multirow}
\usepackage[dvipsnames]{xcolor}

% Used for displaying a sample figure. If possible, figure files should
% be included in EPS format.
%
% If you use the hyperref package, please uncomment the following line
% to display URLs in blue roman font according to Springer's eBook style:
% \renewcommand\UrlFont{\color{blue}\rmfamily}

\begin{document}
	%
	\title{Bidirectional Transformations in Specifications for Runtime Verification}
	%
	%\titlerunning{Abbreviated paper title}
	% If the paper title is too long for the running head, you can set
	% an abbreviated paper title here
	%
	\author{Lisandra Silva}
	%
	% First names are abbreviated in the running head.
	% If there are more than two authors, 'et al.' is used.
	%
	\institute{National Institute of Informatics, Tokyo Japan\\}
	%
	\maketitle              % typeset the header of the contribution
	
	\begin{abstract}
		There are a few different specifications languages to specify monitors for Runtime Verification, and reason about the same property in different specification languages is not straightforward. The present work uses Bidirectional Transformations to synchronize different specifications for Runtime Verification. More specifically to synchronize a specification expressed as a Regular Expression and another expressed as a Finite State Machine.
	
		\keywords{Bidirectional Transformations  \and Runtime Verification \and Regular Expressions \and Non-deterministic Finite Automata \and Deterministic Finite Automata}
	\end{abstract}
	%
	%
	%
	\section{Introduction}
Runtime Verification (RV) is a lightweight formal method to ensure at execution time that a system meets a desirable behaviour. A possible approach for RV consists in analyze an execution trace of the system under scrutiny using a decision procedure called \textit{monitor}. The desirable behaviour of the system can be specified as a set of properties to be verified. From each property a monitor is generated, which primary goal is to detect violation or satisfaction with respect to the given specification, emitting verdicts (truth values) indicating satisfaction or violation of the property ~\cite{rv2,rv3,rvart}.  

There a few different specifications languages to specify monitors for RV, and comparing the expressiveness is these languages is not straightforward. Besides, sometimes it is not easy to select a specific language to write a specification, because the syntax and operations of a particular language may make certain specifications easier or harder to write and/or read ~\cite{rv2,rvart}. 

Some work has being done in translation between languages, for instance the translation of first-order temporal logic into quantified event automata. However, the present work will focus on the translation between Regular Expressions (RE) and Finite State Machine (FSM), two different specification languages to write the RV properties.

There are already algorithms to convert between RE and FSM, and they will be summarized in the present work. However, the algorithms to go from RE to FSM and the one to go from FSM to RE are not inverse. This means that given a RE we can obtain its corresponding FSM, but going back to RE can output a completely different RE. 

When reasoning about the translation of properties between different language specifications would be useful that small changes in one resulted in small changes in another. For a simple example, given the RE (ab*$|$ba*), when one editing on the corresponding FSM just one of the branches of the RE, then when coming back again to the RE only that branch should have been updated. 

\begin{center}
    \textit{``The art of progress is to preserve order amid change and to preserve change amid order''}
\end{center}

\begin{flushright}
    A N Whitehead
\end{flushright}

This is the main motivation to use Bidirectional Transformations (BX) to translate between RE and FSM. BX provide a mechanism to maintain consistency between two pieces of data, the \textit{source} and the \textit{view}. The \textit{get} function extracts part of the information from the source and produces a view, while the \textit{put} function takes the original source and a (possibly) updated view and produces the updated source. The BX must satisfy well-behavedness rules such as \textit{putting} an unmodified view into a source must produce the original source. But besides that, we want to explore the least-changes principle, which states that ``small'' changes on the view lead to ``small'' changes on the source, a topic which is still unsettled and actively investigated by the BX community.




	\par
	
        \section{Runtime Verification}
    \textit{Runtime Verification} (RV) is a dynamic analysis method aiming at checking whether a run of the system under scrutiny satisfies a given correctness property.

  \begin{figure}
    \centering
    \includegraphics [scale=0.4]{Images/rv.png}
    \caption{An overview of the RV process}
    \label{rv}
\end{figure}

 The components of a RV environment are the system to be checked and a set of properties to be checked against the system execution. Properties can be expressed in a formal specification language, or even as a program in a general-purpose programming language. From a given property, a monitor is generated, i.e., a decision procedure for that property - \textit{monitor synthesis}. The system is instrumented to generate the relevant events to be fed into the monitor - \textit{system instrumentation}. In the next step, the system's execution is analyzed by the monitor - \textit{execution analysis}. The monitor is able to consume the events produced by the running system and, for each consumed event, emits a verdict indicating the status of the property, depending on the event sequence seen so far. Finally, the monitor sends feedback to the system so that more specific corrective actions can be taken (Figure \ref{rv}) ~\cite{rvart}.
 
 
\subsection{Formal Specification of the system behaviour}
There are different formal approaches to describe the expected behaviour of a system. But before presenting some different specification languages, let's start by presenting some general properties of these formalisms. 

\subsubsection{Events}
The behaviour of a system can be analyzed as the way the system changes over time, and this can be done through its observation. To abstract these observations we will use \textit{Events} - discrete atomic entities that represent \textit{actions} or \textit{state changes} made by the system. 
The system's observable events of interest is called its \textit{alphabet}. The choice of events is part of the specification and will determine the available information about the system and about which properties can be described.

\subsubsection{Traces}
A \textit{trace} is a sequence of events and abstracts the behaviour of a single run of the system. Obviously, an observable trace must be finite, but it is sometimes useful to think about the possible infinite behaviours of a system. Therefore, a trace can be viewed as a finite prefix of the infinite behaviour.

\subsubsection{Properties and Specifications}
The abstraction of a property can be described as a set of traces and its specification is a concrete (textual) object that denotes this set of traces.
As there are many specifications languages, one can have many specifications for a single property, but a property is unique and independent of the specification language. If the specification language is ambiguous, then the specific property may not be clear. Dealing with such ambiguities is a common issue in the specification process ~\cite{rv2}.


\subsection{Different language specifications}
This section presents a brief description of the main specification languages for RV, as well as some of their general features.

A RV specification language can be \textit{Executable} if the specification is directly executable and therefore more low-level, or \textit{Declarative} in which an executable object (monitor) is generated from the specification. Also, some specification languages are more suited to specify sets of finite traces whereas others are more suited to specify infinite traces. A specification may also capture \textit{good} or \textit{bad} behaviour. A match against a good behaviour specification represents \textit{validation} of the desired property, whereas a match a bad behaviour specification represents \textit{violation} of that property ~\cite{rv2}.

\subsubsection{Regular Expressions}
Regular Expressions are a commonly used formalism in Computer Science for describing sets of strings, but can also be used in RV to describe traces of events, where the atoms are not characters but events. The regular expression matches if any \textit{suffix} of the trace matches the expression - \textit{suffix-matching} - to attest the validation or violation of the property (e.g., in the work of TraceMatches) ~\cite{rvart}.

\subsubsection{Finite State Machines}
Finite State machines have the same expressive power of Regular Expressions but have the advantage of being directly executable, in opposition to regular expressions and temporal logic, since both require \textit{monitor synthesis} to produce a monitor, which is usually described as some form of state machine. 

\vspace{5mm}

The formalisms of \textbf{Linear Temporal Logic} and \textbf{Context Free Grammars} can also be used for specifications in RV, but since they are not the main focus of the present work they will not be further approached. 

    
    \section{Bidirectional Transformations}
\label{chapter:BX}

Bidirectional Transformations (BX) provide a mechanism for maintaining consistency between two pieces of data. A bidirectional transformation consists of a pair of functions: a \textit{get} function that may discard part of information from a source to produce a view, and a \textit{put} function that accepts a source and a possible modified view, and reflects the changes in the source producing an updated source that is consistent with that view. The pair of functions should be \textit{well-behaved} satisfying the next laws:

\begin{flalign*}
    put\ s\ (get\ s) = s \quad (GetPut) \\
    get\ (put\ s\ v)\ = v \quad (PutGet)
\end{flalign*}

The \textit{GetPut} law requires that if no changes are made on the view \textit{get s} then the \textit{put} function should produce the same unmodified source $s$. The \textit{PutGet} law says that all the changes in the view should be reflected in the source, as so getting from an updated source computed by \textit{put}ting a view $v$ should retrieve the same $v$ ~\cite{bigul,brul}.

The straightforward approach to write a bidirectional transformation is to write two separate functions in any language and show that they satisfy the \textit{well-behavedness} rules. Although this ad hoc solution provides full control over both \textit{get} and \textit{put} transformations, verifying that they are \textit{well-behaved} requires intricate reasoning about their behaviours. Besides, any modification to one of the transformations requires a redefinition of the other transformation as well as a new \textit{well-behavedness} proof ~\cite{bigul,boomerang}. 

A better alternative is to write a single program where both transformations can be described at the same time, using for that a \textit{bidirectional programming language}. Many bidirectional languages have been proposed during the past years and the design challenge for all of them lies in striking a balance between expressiveness and reliability. 

There are two approaches when designing bidirectional programming languages: 

\begin{itemize}
    \item \textit{get-based} - allows the programmer to write the \textit{get} function, deriving a suitable \textit{putback} function. The problem with this approach is that the \textit{get} function may not be injective and so there may exist many possible put functions that can be combined with it to form a valid BX.
    
    \item \textit{putback-based} - allows the programmer to write the \textit{put} from which the only \text{get} function is automatically  derived. The resulting pair of functions must form a \textit{well-behaved} bidirectional transformation.
\end{itemize}

Despite this, as the present work is still exploratory, it focused on studying the possibility of implement a BX between RE and DFA with the first alternative, where the two functions \textit{get} and \textit{put} are written separately. 
	
    \section{Regular Expressions and Finite State Machine}

Regular expressions and Finite Automata (FA) represent a valuable concept in theoretical computer science. Their equivalence is well known dating back to the Kleen's paper in 1956. REs are well suited for human users and therefore often used as interfaces to specify certain patterns or languages, while automata immediately translate to efficient data structures and are suited for programming tasks. Obviously this fact raises the interest in conversion algorithms between them ~\cite{refa}. 

Moreover, when reasoning about the specifications of properties for RV, it can be useful to convert between them and observe how changes in one of the specifications are reflected in the other. That was the main motivation for use Bidirectional Transformations between these two types of specifications.

The next sections will describe some formal definitions of RE and FA as well as the known algorithms to convert between them. 

\subsection{Formal definitions}
\label{definitions}
Before introduce the known algorithms for conversion it is useful to present and formally describe the two types of automata. 

\subsubsection{Regular Expressions} 
Given a vocabulary $\Sigma$, a symbol $a \in \Sigma$ and the regular expressions $s$ and $t$, then RE can be defined inductively as following: 

\begin{equation*}
    RE = \quad \emptyset \quad | \quad \epsilon \quad | \quad a \quad | \quad (s+t) \quad | \quad (s\ .\ t) \quad | \quad (s)^*
\end{equation*}

The language defined by a regular expression $r$, denoted by $L(r)$, is defined as follows:

\begin{flalign*}
    L(\emptyset) & = \emptyset \\
    L(\epsilon) & = \{\epsilon\} \\
    L(a) & = \{a\} \\
    L(s+t) & = L(s) \cup L(t) \\
    L(s\ .\ t) & = L(s)\ .\ L(t) \\
    L(s^*) & = L(s)^*
\end{flalign*}

\subsubsection{Non-Deterministic Finite Automata (NDFA)} Formally, a NDFA is described as follows:  

\begin{equation*}
    A = (Q,\Sigma,q_0,F,\delta)
\end{equation*}

where, $Q$ is the finite set of states, $\Sigma$ is the finite set of input symbols (the vocabulary), $q_0 \in Q$ is the initial states, $F \in Q$ is the set of accepting (final) states, and $\delta : Q \times (\Sigma \cup \{\epsilon\}) \rightarrow 2^Q$ is the \textit{transition function}. 

As the name itself pronounces, the NDFA can transit to one or more states when consuming a given symbol. The next state is computed as the set of all possible states to which is possible to transit. The NDFA can have \textit{$\epsilon$-transitions}, transitions without consuming any input symbol, i.e. the empty string $\epsilon$ is a possible input. 
If the NDFA has no $\epsilon$-transitions, i.e. is $\epsilon$-free, then the transition function can be restricted to $\delta : Q \times \Sigma \rightarrow 2^Q$.

\subsubsection{Deterministic Finite Automata (DFA)} \textit{Deterministic} refers to the uniqueness of the computation. It can be seen as a special kind of NDFA in which for each state and symbol, the result of the transition has exactly one state. Following is the formal definition of a DFA:

\begin{equation*}
    A = (Q,\Sigma,q_0,F,\delta)
\end{equation*}

where, $Q$ is the finite set of states, $\Sigma$ is the finite set of input symbols (the vocabulary), $q_0 \in Q$ is the initial state, $F \in Q$ is the set of accepting (final) states, and $\delta : Q \times \Sigma \rightarrow Q$ is the \textit{transition function}. 

\subsection{Conversion algorithms}
This sections recalls the most prominent algorithms for conversion from a RE to a DFA. 

\subsubsection{Thompson's construction algorithm}
Converts a RE into an equivalent NDFA. The algorithm splits an expression into its constituent sub-expressions and works recursively applying the next rules: 

\begin{itemize}
    \item  $\bf e = \emptyset$ is converted in:
    \begin{figure}[H]
        \begin{center}
        \resizebox{.3\textwidth}{!}{\includegraphics{Images/NdfaREempty.pdf}}
        \end{center} 
    \end{figure}
    
    \item $\bf e = \epsilon$  is converted in:
    \begin{figure}[H]
        \begin{center}
        \resizebox{.1\textwidth}{!}{\includegraphics{Images/NdfaREepsilon.pdf}}
        \end{center} 
    \end{figure}
    
    \item  $\bf e = a$ is converted in:
    \begin{figure}[H]
        \begin{center}
        \resizebox{.3\textwidth}{!}{\includegraphics{Images/NdfaRElit.pdf}}
        \end{center} 
    \end{figure}
    
    \item  $\bf e = p\ .\ q$ is converted in:
    \begin{figure}[H]
        \begin{center}
        \resizebox{.7\textwidth}{!}{\includegraphics{Images/NdfaREthen.pdf}}
        \end{center} 
    \end{figure}
    
    \item  $\bf e = p + q$ is converted in:
    \begin{figure}[H]
        \begin{center}
        \resizebox{.6\textwidth}{!}{\includegraphics{Images/NdfaREor.pdf}}
        \end{center} 
    \end{figure}
    
    \item  $\bf e = p^*$ is converted in:
    \begin{figure}[H]
        \begin{center}
        \resizebox{.6\textwidth}{!}{\includegraphics{Images/NdfaREstar.pdf}}
        \end{center} 
    \end{figure}
\end{itemize}

For instance, the regular expression $\bf e = (\epsilon + a^*b)$ would be converted in the NDFA presented in Figure \ref{fig:ndfaT}

\begin{figure}
    \centering
    \resizebox{.8\textwidth}{!}{\includegraphics{Images/ndfa.png}}
    \caption{NDFA resulting from $\bf e = (\epsilon + a^*b)$ using Thompson's construction }
    \label{fig:ndfaT}
\end{figure}


\subsubsection{Glushkov's construction algorithm}
Is another well-known algorithm to convert a give RE into a NDFA. Is similar to Thompson's construction, once the $\epsilon$-transitions are removed. Following are the construction steps to create a NDFA that accepts the language $L(e)$ accepted by the RE $e$:

\begin{itemize}
    \item \textbf{Step 1} - linearisation of the expression. Each letter of the alphabet appearing in the expression e is renamed, so that each letter occurs at most once in the new expression $e'$. Let $A$ be the old alphabet and let $B$ be the new one;
    \item \textbf{Step 2a} - the following sets are computed:
    \begin{itemize}
        \item $P(e') = \{\ x \in B\quad |\quad xB^* \cap L(e') \neq \emptyset\ \}$  is the set of symbols which occurs as first letter of a word of $L(e')$. Can be defined inductively:
        
        \begin{flalign*}
            P(\emptyset) & = P(\epsilon) = \emptyset \\
            P(a) & = \{a\} \text{, for each letter $a$} \\
            P(s+t) & = P(s) \cup P(t) \\
            P(s\ .\ t) & = P(s)\ \cup \ \Lambda (s) P(t) \\
            P(s^*) & = P(s)
        \end{flalign*}
        
        
        \item $D(e') = \{\ y \in B\quad |\quad B^*y \cap L(e') \neq \emptyset\ \}$ is the set of symbols that can end a word of $L(e')$. Can be defined inductively with the same rules of $P$ except for the product where
        
        \begin{equation*}
            D(s\ .\ t) = D(t)\ \cup \ D(s)\Lambda(t) 
        \end{equation*}
        
        \item $F(e') = \{\ u \in B^2\quad |\quad B^*uB^* \cap L(e') \neq \emptyset\ \}$ is the set f symbol pair that can occur in words of $L(e')$. Can be defined inductively as follows: 
        \begin{flalign*}
            F(\emptyset) & = F(\epsilon) = F(a) = \emptyset \text{, for each letter $a$} \\
            F(s+t) & = F(s) \cup F(t) \\
            F(s\ .\ t) & = F(s)\ \cup\ F(t)\ \cup\ D(s)P(t) \\
            F(s^*) & = F(s)\ \cup\ D(e)P(e)
        \end{flalign*}
        
    \end{itemize}
    
    \item \textbf{Step 2b} - computes the set $\Lambda = \{\epsilon\} \cup L(e')$, in case the empty word belongs to the language, otherwise $\Lambda = \emptyset$. It can be defined inductively for each RE as following:
    \begin{flalign*}
        \Lambda(\emptyset) & = \emptyset \\
        \Lambda(\epsilon) & = \{\epsilon\} \\
        \Lambda(a) & = \emptyset \text{, for each letter $a$} \\
        \Lambda(s+t) & = \Lambda(s) \cup \Lambda(t) \\
        \Lambda(s\ .\ t) & = \Lambda(s)\ .\ \Lambda(t) \\
        \Lambda(s^*) & = \{\epsilon\} 
    \end{flalign*}
    
    
    \item \textbf{Step 3} - computation of the local language, i.e. the set of words which begin with a letter of $P$, end by a letter of $D$ and whose transitions belong to $F$: 
    \begin{equation*}
        L' = (PB^* \cap B^*D)\ \backslash\ B^* (B^2 \backslash F) B^*
    \end{equation*}
    
    \item \textbf{Step 4} - erasing the delinearization, giving to each letter of $B$ the letter of $A$.
\end{itemize}

Given the regular expression $\bf e = (a(ab)^*)^* + (ba)^*$, would produce the NDFA in Figure \ref{fig:ndfaG}, without performing Step 4, to perform this step one only needs to remove the index from each symbol in the nodes.

\begin{figure}
    \centering
    \resizebox{.8\textwidth}{!}{\includegraphics{Images/glushkov.jpg}}
    \caption{NDFA resulting from $\bf e = (a(ab)^*)^* + (ba)^*$ using Glushkov's construction }
    \label{fig:ndfaG}
\end{figure}

\subsubsection{Powerset construction algorithm}
Converts a NDFA into a DFA, which recognize the same formal language. Recall the structure of a NDFA in Section \ref{definitions}, the corresponding DFA has the set of states $Q_D$, where each state corresponds to subsets of $Q_N$. 

The initial state of the DFA is the set $\{q_0\}$ where $q_0$ is the initial state of the NDFA. In case of a NDFA has $\epsilon$-transitions then the initial state of the DFA - $q_{0D}$ - has the initial state of the NDFA - $q_{0N}$ - plus the states reachable from that state through $\epsilon$-transitions - called $\epsilon$-\textit{closure}:
\begin{equation*}
    q_{0D} = q_{0N}\ \cup \ \epsilon\text{-\textit{closure}} \ \{q_{0N}\}
\end{equation*}

Starting from the initial state $q_{0D}$, each new state is computed from another state $S \in Q_D$. It can be defined recursively as:
\begin{itemize}
    \item $q_{0D} \in Q_D$
    \item $qs \in Q_D \Rightarrow \{ d\ |\ (o,y,d) \in \delta_{NDFA} \wedge o \in qs\} \in Q_Dv , \quad \forall y \in \Sigma$
\end{itemize}

The transition function of the DFA maps a state $S$ and the input symbol $y$ to the respective computed state:

\begin{equation*}
    \delta_{DFA} = \cup \{(S,y,D) | D = \{ d\ |\ (o,y,d) \in \delta_{NDFA} \wedge o \in S\}\}, \forall{(y \in \Sigma \wedge S \in Q_D)}
\end{equation*}

Finally, the set of accepting states of the DFA $Z_D$ is the set of states $S \in Q_D$ that contain at least one state that is accepting state in the NDFA:

\begin{equation*}
    Z_D = \{q \in Q_D\ |\ q \cap Z_N \neq \emptyset\}
\end{equation*}

\subsubsection{Kleene's Algorithm}

Kleene's algorithm produces a regular expression from a DFA in the following way.
\begin{enumerate}
    \item It first constructs a graph whose nodes correspond to the states of the automaton and the edges are regular expressions that connect those nodes. 
    \item The initial set of edges contains one edge for each transition in the automaton (labelled with the corresponding literal) plus one edge for each node to itself containing the regular expression $\epsilon$.
    \item It proceeds by applying a variation of the Floyd-Warshall algorithm in order to obtain all paths connecting each of the nodes (summing weights corresponds to concatenation of regular expressions whereas the minimum operation corresponds to alternative ($+$) of regular expressions).
    \item Finally the regular expression equivalent to the original DFA can be obtained as the union ($+$) of the edges which connect the initial state to all final states.
\end{enumerate}
	
    \section{Implementation}
This section will describe the part of the implementation of the Bidirectional Transformations between a RE and a DFA. It will start by describing the reasoning about how to implement the BX and will follow by explaining the implementation process, including some rollback along the way. Finally, it will present the \textit{put} strategy defined for a BX between NDFA and DFA.

The source code of the implementation is available in this Github repository:  \url{https://github.com/lisandrasilva/rv-bx}.

\subsection{Reasoning about BX}
Considering the already existent algorithms to convert between a RE and a DFA, we decided to attempt to implement the BX as a composition of two small BX, more concretely, one between RE and NDFA and another between NDFA and DFA. As depicted in Figure \ref{fig:BX}) the NDFA would be an intermediate structure used as a view in one of the BX and as a source in the another one.  

\begin{figure}
    \centering
    \resizebox{.4\textwidth}{!}{\includegraphics{Images/BX_arc.png}}
    \caption{BX between RE and DFA}
    \label{fig:BX}
\end{figure}

In the BX between RE and NDFA the source is the RE and the view is the NDFA. The \textit{getNDFA} function from RE to NDFA would be one of the conversion algorithms presented on Section \ref{convert} - Thompson's construction or Glushkov's construction. Regarding the BX between NDFA and the DFA, the source is the NDFA and the view is the DFA, where the \textit{getDFA} function would be the Powerset construction algorithm.

\vspace{5mm}
The \textit{get} function that given a RE as source and produces a DFA as view will be the composition of these two functions:

\begin{center}
    \textit{\textbf{get re} = (getDFA . getNDFA) re}
\end{center}

\vspace{5mm}
The main goal is to define the \textit{putback} functions:

\vspace{3mm}
\textbf{\textit{putNDFA}} - function that accepts a NDFA as a source and a possible modified DFA as a view and reflects the changes on the view into the source;
    
\textbf{\textit{putRE}} - function that accepts a RE as a source and a possible modified NDFA as a view and reflects the changes on the view into the source.

\vspace{5mm}
The final \textit{putback} function that accepts a RE as a source and a possible modified DFA as a view and produces the updated RE will be the composition of these two functions:


\begin{center}
    \textit{\textbf{put re dfa} = putRE re (putNDFA (get re) dfa)}
\end{center}

To tackle the problem we decided to start by define the \textit{putNDFA} function.

\subsection{Get function}
As said above the \textit{get} function will be the composition of two, with a NDFA as an intermediate structure. Once this NDFA will be an argument of the \textit{putNDFA} function as the source and \textit{putRE} function as the view, it is necessary to reason about what type of NDFA we should consider, which means reason about the choice for the \textit{getNDFA} function.

The NDFA can be obtained through the Thompson's construction algorithm or through the Glushkov's construction algorithm. But, recall that the resulting NDFA are different since the one obtained through Glushkov's algorithm is \textit{$\epsilon$-free}, while the one obtained through the Thompson's algorithm has $\epsilon$-transitions. 

\subsubsection{NDFA with $\epsilon$-transitions}
Let's consider the NDFA in Figure \ref{fig:epsilon} and its correspondent DFA in Figure \ref{fig:epsilondfa} obtained through the Powerset construction algorithm. 

\begin{figure}
    \centering
    \resizebox{.7\textwidth}{!}{\includegraphics{Images/epsilondfa.png}}
    \caption{NDFA with $\epsilon$-transitions}
    \label{fig:epsilon}
\end{figure}

\begin{figure}
    \centering
    \resizebox{.7\textwidth}{!}{\includegraphics{Images/epsilon_DFA.png}}
    \caption{DFA obtained from NDFA in Figure \ref{fig:epsilon}}
    \label{fig:epsilondfa}
\end{figure}

By analyzing  the pictures the pictures is very hard to reason about a \textit{putback} function when the NDFA has $\epsilon$-transitions. For instance, in the obtained DFA of the given example (Figure \ref{fig:epsilondfa}) all the nodes are dependent on the transitions of the 'C' node in the NDFA, which means that when adding or removing transitions from or to node 'C' in the NDFA, this changes will be reflected in all the nodes with 'C' in the DFA (because that is what is demanded by the Powerset construction).

This fact results in making the possible changes in the view very restrictive, even though we designed a pseudo-algorithm to \textit{put} function in many cases it didn't satisfied the GetPut law. 

\subsubsection{NDFA \textit{$\epsilon$-free}} Figure \ref{fig:glundfa} depicts a NDFA \textit{$\epsilon$-free} equivalent to the NDFA in Figure \ref{fig:epsilondfa}. In Figure \ref{fig:gludfa} is the correspondent DFA. 

\begin{figure}[H]
    \centering
    \resizebox{.8\textwidth}{!}{\includegraphics{Images/Glushkovdot.png}}
    \caption{NDFA \textit{$\epsilon$-free}}
    \label{fig:glundfa}
\end{figure}

\begin{figure}
    \centering
    \resizebox{.8\textwidth}{!}{\includegraphics{Images/gluDFAdot.png}}
    \caption{DFA obtained from NDFA in Figure \ref{fig:epsilon}}
    \label{fig:gludfa}
\end{figure}

As we can conclude by analyzing the two automata the dependencies here are less than in the other one, just between nodes \texttt{[a1,a2]} and \texttt{[a2]}, because both have the node \texttt{a2} from the NDFA, and possibly changes in this node will be reflected in both nodes in the view, even only one was modified in the first place. 

However, as it allows a wider range of possible changes on the view we decided to choose the \textit{$\epsilon$-free} NDFA to define the \textit{put} strategy between NDFA and DFA, which means that the \textit{get} function from RE to NDFA should be the Glushkov's construction algorithm. 

\subsection{Put back from DFA to NDFA}

\subsubsection{Structures for DFA and NDFA}
When reasoning about the put back function, the first step was to define the structures to represent both the NDFA and the DFA. Both structures are structurally very similar, but they will differ in the \texttt{delta} function, since in the DFA is not allowed to have two transitions from the same node with the same symbol, while in the NDFA that is possible. 

\begin{verbatim}
data Ndfa st sy = Ndfa { vocabularyN :: [sy ],
                         statesN     :: [st ],
                         initialSN   :: [st ],
                         finalSN     :: [st ],
                         deltaN      ::[((st,sy),st)]
                       }
                   
data Dfa st sy = Dfa { vocabularyD ::[sy],
                       statesD     ::[st],
                       initialSD   :: st,
                       finalSD     ::[st],
                       deltaD      ::[((st,sy),st)]
                     }
\end{verbatim}

We decided to represent the structures with polymorphism on the arguments so they can be more flexible on the types they can represent. However, remind that in the present work, the DFA is result of the Powerset construction of the NDFA, which means that the states in the DFA are sets of the states in the NDFA. As so, for instance, if in NDFA the type of \texttt{st} is \texttt{Int}, then the type of \texttt{st} in the DFA is \texttt{[Int]}.

\subsubsection{Well built DFA}
As the DFA can be edited by the user, and this changes will be reflected in the NDFA, it is important to guarantee that the DFA is well built before proceed with the \textit{put} function. This is guaranteed by the function \texttt{wellBuilt} that ensures the following:

\begin{itemize}
    \item Every transition from a given origin $o$ with a symbol $s$ is unique;
    \item Every Node $x$ must be reachable from the start node;
    \item Every node must reach an accepting node;
    \item For every transition $o \xrightarrow{s} d $, each state in the destination's list of $d$ states must be prefixed by the symbol $s$ (considering the NDFA is result of Glushkov's algorithm and DFA is result of Powerset construction);
    \item The initial state cannot be modified (also a requirement of Glushkov's algorithm)
\end{itemize}

The function returns an error message for each one of the above mentioned rules in case they are violated. 
    
\subsubsection{Reflecting transition table}
The function \texttt{getTable} is responsible for reflecting the DFA transition table into the NDFA transition table. Therefore, it receives as argument the NDFA transition table, the DFA transition table zipped with an accumulator that will save the subset of $d$ of the NDFA transitions that were validated so far, it will also receive the set of states $q$ of the DFA to perform checks and returns the updated NDFA transition table. 

%Each transition \texttt{((o,s),d)} of the DFA is zipped with an accumulator that will save the subset of $d$ of the NDFA transitions that were validated so far. %Thus the DFA table transition received by the \texttt{getTable} function is in fact \texttt{[((([st],sy),[st]),[st])]}. 

\vspace{5mm}
\begin{verbatim}
getTable :: (Ord st, Ord sy) =>[((st,sy),st)] 
                             ->[((([st],sy),[st]),[st)]
                             -> [[st]] 
                             ->[((st,sy),st)]
getTable [] dfaT q = 
    let toAdd = filter (not . null . snd)
                [(((os,d),ds\\rs)) |(((os,d),ds),rs) <- dfaT]
    in concat $ map (`rearrangeS' q)toAdd
    
getTable (t@((o,s),d):ts) dfaT q = 
    let relS = [ x | x <- q , o `elem' x]
        trns = [ 1 | (((os,sym),ds),rs)<- dfaT , 
                                          os `elem' relS ,
                                          sym == s , 
                                          d `elem' ds]
        dfaT' = map (updateListAux t)dfaT
    in if (length relS == length trns) &&(not $ null relS) 
       then t:getTable ts dfaT' q
       else getTable ts dfaT q
\end{verbatim}

\vspace{5mm}
%The function consumes each NDFA transition table recursively. 
For each NDFA transition \texttt{((o,s),d)}, the function verifies that each node in the DFA that contains `$o$' has the transition through symbol `$s$' for a node that contains `$d$'. When this is valid then the NDFA transition is kept in the source and the `$d$' is added to the auxiliary list from the DFA transitions mentioned above (using \texttt{updateListAux} function), otherwise is discarded. 

\begin{verbatim}
updateListAux ((o,s1),d)(((os,s2),ds),rs) = 
            if o `elem' os && s1 == s2 &&d `elem' ds
            then (((os,s2),ds),d:rs)
            else (((os,s2),ds),rs)
\end{verbatim}

When the NDFA transition table is empty it means that all of the old transitions (if existent) were already validated or discarded. At this point, for each DFA transition \texttt{((os,s),ds)} its corresponding auxiliary list with the subset of $ds$ that were already validated is removed from the set $ds$. If the remaining $ds$ list is empty then it means that the transition was already validated. When there are still states in the $ds$ list the new transitions in the NDFA must be created. This is done in the \texttt{rearrangeS} function. 

\begin{verbatim}
rearrangeS :: Eq st => (([st], sy), [st])
                    -> [[st]] 
                    -> [((st, sy), st)]
rearrangeS ((os,sy),dsts) q = 
    let min = [((o,sy),dd) | o <- os \\ (concat $ delete os q), 
                             dd <- dsts]
    in if null min then [((o,sy),dd) | o <- os, dd <- dsts]
       else min 
\end{verbatim}

The function receives a DFA transition, the set of DFA states and creates the list of new NDFA transitions corresponding to the given DFA transition. The \texttt{min} is the list of NDFA transitions in which the NDFA origin nodes that appear in other DFA nodes are removed. 

For instance, taking the example in Figures \ref{fig:glundfa} and \ref{fig:gludfa}, if a new DFA transition is added (([a1,a2],c),[c1]) only the NDFA transition ((a1,c),c1) must be created since if the transition ((a2,c),c1) is also created then when \textit{getting} again the view the DFA would also have the edge (([a2],c),[c1]), that was not created by the user, and so the GetPut law would not be satisfied. When the computed \texttt{min} list is empty it means that is not possible to create new edges without interfere with other DFA nodes, that is called an inconsistency with the source. In this case the function returns all possible NDFA edges for the respective DFA transition, and the inconsistency is dealed with later. 

\subsubsection{Inconsistencies}
Let's take a look at the NDFA and DFA in the Figures \ref{fig:glundfa} and \ref{fig:gludfa} respectively. The NDFA has the transition \texttt{((a2,b),b3)}, which results in the corresponding DFA to have the transitions \texttt{(([a1,a2],b),[b3])} and \texttt{(([a2],b),[b3])}. This strategy allows for instance the user to remove the transition \texttt{(([a2],b),[b3])}, without removing the transition \texttt{(([a1,a2],b),[b3])}, because this transition can exist due to the transition \texttt{((a1,b),b3)} in the NDFA. However, it doesn't allow the user to remove the edge \texttt{(([a1,a2],b),[b3])} without removing the edge \texttt{(([a2],b),[b3])}, since the first is dependent on the second, here we have again an inconsistency with the source, and the second edge is also removed.

When an inconsistency occurs in the view, it means that is not possible to get the same view with the updated source, because the dependencies in the view were not `respected'. In this cases, the user is asked if he wants to proceed with the consistent version of the view, and in that case the source is updated with the consistent version of the view.

\subsubsection{Put Function}
After the table transition of the NDFA is complete it is still necessary to update the other fields of the NDFA structure, this is done by the function \texttt{putNdfaStruct}.

\begin{verbatim}
putNdfaStruct :: (Ord st, Ord sy) => Ndfa st sy  
                                  -> Dfa [st] sy 
                                  -> Ndfa st sy
putNdfaStruct (Ndfa v1 q1 s1 z1 d1) (Dfa v2 q2 s2 z2 d2) = 
        Ndfa v q s z d
    where v = v2
          q = nub $ concat q2
          s = s1
          z = f z1 (concat z2)
          d = getTable d1 (zip d2 (repeat [])) q2
          f [] cz2 = cz2
          f (h:t) cz2 = if h `elem` cz2 
                        then h:(f t (filter (/= h) cz2))
                        else f t cz2
\end{verbatim}

The put of the other fields of the structure is very straight, the most complex is the final states. For that, for each old final state the function tests if it still belongs to some of the new final states and in that case it is kept as a final state or discarded otherwise. In the end the remaining final states are added as final state. 


Finally, the function \texttt{putNDFA} puts it all together: it receives the NDFA, the (possibly) updated DFA and returns an \texttt{Error NDFA}, which is a data type that can contain a NDFA if everything went well or an error message in case the function \texttt{wellBuilt} has returned some error. 

\begin{verbatim}
putNdfa :: Ndfa (Indexed Char) Char 
        -> Dfa [Indexed Char] Char
        -> Error (Ndfa (Indexed Char) Char)
putNdfa ndfa dfa = case wellBuilt dfa of 
                        Ok True -> Ok (putNdfaStruct ndfa dfa)
                        Error m -> Error m
\end{verbatim}

It was also created an interpreter to interact with the user, the interpreter calls the \texttt{putNdfa} function after the user edits the view, and in case the view is not consistent with source asks the user if he wants to continue or to rollback, in case he wants to continue the source and the view are replaced for the consistent versions.
    
    \section{Conclusions and Future work}
The initial goal was to implement the BX as a composition of two BX's, one between RE and NDFA and another between NDFA and DFA. However, only the second one was defined. As future work it is still necessary to implement the first one, between RE and NDFA, considering as the \textit{get} function the Glushkov's construction algorithm. 

As said in Section \ref{chapter:BX}, this was an exploratory work that used the `naive' way to implement a BX where the two functions \textit{get} and \textit{put} separately. To (cleverly) avoid the need of a formal proof that the two functions are \textit{well-behaved} it would be very useful to use a \textit{bidirectional programming} language to implement the put algorithm. We suggest a \textit{putback based} approach, more concretely BiGUL.

Finally, some work has being done in translation between specification languages, a challenge for the future could be implement synchronizations between another specification languages, namely Linear Temporal Logic and Context Free Grammars.
	
	
	% ---- Bibliography ----
	%
	% BibTeX users should specify bibliography style 'splncs04'.
	% References will then be sorted and formatted in the correct style.
	%
	\bibliographystyle{apalike}
	\bibliography{mybib}
	%
	
\end{document}

% LLNCS macro package for Springer Computer Science proceedings;
% Version 2.20 of 2017/10/04
%
\documentclass[runningheads]{llncs}
%
\usepackage{graphicx}
\usepackage{lipsum}
\usepackage{amsmath}
\usepackage{authblk}
\usepackage{makecell,array}
%\usepackage{natbib}
%\usepackage[square,sort,comma,numbers]{natbib}
%\usepackage[numbers]{natbib}

\usepackage[square,numbers]{natbib}
\usepackage{graphicx}
\usepackage{float}
\usepackage{gensymb}
\usepackage{wrapfig}
\usepackage{amssymb}
%\setcounter{secnumdepth}{3}
% Used for displaying a sample figure. If possible, figure files should
% be included in EPS format.
%
% If you use the hyperref package, please uncomment the following line
% to display URLs in blue roman font according to Springer's eBook style:
% \renewcommand\UrlFont{\color{blue}\rmfamily}

\begin{document}
	%
	\title{Bidirectional Transformations between specifications languages for Runtime Verification}
	%
	%\titlerunning{Abbreviated paper title}
	% If the paper title is too long for the running head, you can set
	% an abbreviated paper title here
	%
	\author{Lisandra Silva a73559@uminho.pt}
	%
	% First names are abbreviated in the running head.
	% If there are more than two authors, 'et al.' is used.
	%
	\institute{National Institute of Informatics, Tokyo Japan\\}
	%
	\maketitle              % typeset the header of the contribution
	
	\begin{abstract}
		Use of Bidirectional Transformations to synchronize different specifications for Runtime Verification. More specifically to synchronize a specification expressed as a Regular Expression and another expressed as a Finite State Machine.
	
		\keywords{Bidirectional Transformations  \and Runtime Verification \and Regular Expressions \and Non-deterministic Finite Automata \and Deterministic Finite Automata}
	\end{abstract}
	%
	%
	%
	\section{Introduction}
Runtime Verification (RV) is a lightweight formal method to ensure at execution time that a system meets a desirable behaviour. A possible approach for RV consists in analyze an execution trace of the system under scrutiny using a decision procedure called \textit{monitor}. The desirable behaviour of the system can be specified as a set of properties to be verified. From each property a monitor is generated, which primary goal is to detect violation or satisfaction with respect to the given specification, emitting verdicts (truth values) indicating satisfaction or violation of the property ~\cite{rv2,rv3,rvart}.  

There a few different specifications languages to specify monitors for RV, and comparing the expressiveness is these languages is not straightforward. Besides, sometimes it is not easy to select a specific language to write a specification, because the syntax and operations of a particular language may make certain specifications easier or harder to write and/or read ~\cite{rv2,rvart}. 

Some work has being done in translation between languages, for instance the translation of first-order temporal logic into quantified event automata. However, the present work will focus on the translation between Regular Expressions (RE) and Finite State Machine (FSM), two different specification languages to write the RV properties.

There are already algorithms to convert between RE and FSM, and they will be summarized in the present work. However, the algorithms to go from RE to FSM and the one to go from FSM to RE are not inverse. This means that given a RE we can obtain its corresponding FSM, but going back to RE can output a completely different RE. 

When reasoning about the translation of properties between different language specifications would be useful that small changes in one resulted in small changes in another. For a simple example, given the RE (ab*$|$ba*), when one editing on the corresponding FSM just one of the branches of the RE, then when coming back again to the RE only that branch should have been updated. 

\begin{center}
    \textit{``The art of progress is to preserve order amid change and to preserve change amid order''}
\end{center}

\begin{flushright}
    A N Whitehead
\end{flushright}

This is the main motivation to use Bidirectional Transformations (BX) to translate between RE and FSM. BX provide a mechanism to maintain consistency between two pieces of data, the \textit{source} and the \textit{view}. The \textit{get} function extracts part of the information from the source and produces a view, while the \textit{put} function takes the original source and a (possibly) updated view and produces the updated source. The BX must satisfy well-behavedness rules such as \textit{putting} an unmodified view into a source must produce the original source. But besides that, we want to explore the least-changes principle, which states that ``small'' changes on the view lead to ``small'' changes on the source, a topic which is still unsettled and actively investigated by the BX community.




	\par
	
        \section{Runtime Verification}
    \textit{Runtime Verification} (RV) is a dynamic analysis method aiming at checking whether a run of the system under scrutiny satisfies a given correctness property.

  \begin{figure}
    \centering
    \includegraphics [scale=0.4]{Images/rv.png}
    \caption{An overview of the RV process}
    \label{rv}
\end{figure}

 The components of a RV environment are the system to be checked and a set of properties to be checked against the system execution. Properties can be expressed in a formal specification language, or even as a program in a general-purpose programming language. From a given property, a monitor is generated, i.e., a decision procedure for that property - \textit{monitor synthesis}. The system is instrumented to generate the relevant events to be fed into the monitor - \textit{system instrumentation}. In the next step, the system's execution is analyzed by the monitor - \textit{execution analysis}. The monitor is able to consume the events produced by the running system and, for each consumed event, emits a verdict indicating the status of the property, depending on the event sequence seen so far. Finally, the monitor sends feedback to the system so that more specific corrective actions can be taken (Figure \ref{rv}) ~\cite{rvart}.
 
 
\subsection{Formal Specification of the system behaviour}
There are different formal approaches to describe the expected behaviour of a system. But before presenting some different specification languages, let's start by presenting some general properties of these formalisms. 

\subsubsection{Events}
The behaviour of a system can be analyzed as the way the system changes over time, and this can be done through its observation. To abstract these observations we will use \textit{Events} - discrete atomic entities that represent \textit{actions} or \textit{state changes} made by the system. 
The system's observable events of interest is called its \textit{alphabet}. The choice of events is part of the specification and will determine the available information about the system and about which properties can be described.

\subsubsection{Traces}
A \textit{trace} is a sequence of events and abstracts the behaviour of a single run of the system. Obviously, an observable trace must be finite, but it is sometimes useful to think about the possible infinite behaviours of a system. Therefore, a trace can be viewed as a finite prefix of the infinite behaviour.

\subsubsection{Properties and Specifications}
The abstraction of a property can be described as a set of traces and its specification is a concrete (textual) object that denotes this set of traces.
As there are many specifications languages, one can have many specifications for a single property, but a property is unique and independent of the specification language. If the specification language is ambiguous, then the specific property may not be clear. Dealing with such ambiguities is a common issue in the specification process ~\cite{rv2}.


\subsection{Different language specifications}
This section presents a brief description of the main specification languages for RV, as well as some of their general features.

A RV specification language can be \textit{Executable} if the specification is directly executable and therefore more low-level, or \textit{Declarative} in which an executable object (monitor) is generated from the specification. Also, some specification languages are more suited to specify sets of finite traces whereas others are more suited to specify infinite traces. A specification may also capture \textit{good} or \textit{bad} behaviour. A match against a good behaviour specification represents \textit{validation} of the desired property, whereas a match a bad behaviour specification represents \textit{violation} of that property ~\cite{rv2}.

\subsubsection{Regular Expressions}
Regular Expressions are a commonly used formalism in Computer Science for describing sets of strings, but can also be used in RV to describe traces of events, where the atoms are not characters but events. The regular expression matches if any \textit{suffix} of the trace matches the expression - \textit{suffix-matching} - to attest the validation or violation of the property (e.g., in the work of TraceMatches) ~\cite{rvart}.

\subsubsection{Finite State Machines}
Finite State machines have the same expressive power of Regular Expressions but have the advantage of being directly executable, in opposition to regular expressions and temporal logic, since both require \textit{monitor synthesis} to produce a monitor, which is usually described as some form of state machine. 

\vspace{5mm}

The formalisms of \textbf{Linear Temporal Logic} and \textbf{Context Free Grammars} can also be used for specifications in RV, but since they are not the main focus of the present work they will not be further approached. 

	
    \section{Regular Expressions and Finite State Machine}

Regular expressions and Finite Automata (FA) represent a valuable concept in theoretical computer science. Their equivalence is well known dating back to the Kleen's paper in 1956. REs are well suited for human users and therefore often used as interfaces to specify certain patterns or languages, while automata immediately translate to efficient data structures and are suited for programming tasks. Obviously this fact raises the interest in conversion algorithms between them ~\cite{refa}. 

Moreover, when reasoning about the specifications of properties for RV, it can be useful to convert between them and observe how changes in one of the specifications are reflected in the other. That was the main motivation for use Bidirectional Transformations between these two types of specifications.

The next sections will describe some formal definitions of RE and FA as well as the known algorithms to convert between them. 

\subsection{Formal definitions}
\label{definitions}
Before introduce the known algorithms for conversion it is useful to present and formally describe the two types of automata. 

\subsubsection{Regular Expressions} 
Given a vocabulary $\Sigma$, a symbol $a \in \Sigma$ and the regular expressions $s$ and $t$, then RE can be defined inductively as following: 

\begin{equation*}
    RE = \quad \emptyset \quad | \quad \epsilon \quad | \quad a \quad | \quad (s+t) \quad | \quad (s\ .\ t) \quad | \quad (s)^*
\end{equation*}

The language defined by a regular expression $r$, denoted by $L(r)$, is defined as follows:

\begin{flalign*}
    L(\emptyset) & = \emptyset \\
    L(\epsilon) & = \{\epsilon\} \\
    L(a) & = \{a\} \\
    L(s+t) & = L(s) \cup L(t) \\
    L(s\ .\ t) & = L(s)\ .\ L(t) \\
    L(s^*) & = L(s)^*
\end{flalign*}

\subsubsection{Non-Deterministic Finite Automata (NDFA)} Formally, a NDFA is described as follows:  

\begin{equation*}
    A = (Q,\Sigma,q_0,F,\delta)
\end{equation*}

where, $Q$ is the finite set of states, $\Sigma$ is the finite set of input symbols (the vocabulary), $q_0 \in Q$ is the initial states, $F \in Q$ is the set of accepting (final) states, and $\delta : Q \times (\Sigma \cup \{\epsilon\}) \rightarrow 2^Q$ is the \textit{transition function}. 

As the name itself pronounces, the NDFA can transit to one or more states when consuming a given symbol. The next state is computed as the set of all possible states to which is possible to transit. The NDFA can have \textit{$\epsilon$-transitions}, transitions without consuming any input symbol, i.e. the empty string $\epsilon$ is a possible input. 
If the NDFA has no $\epsilon$-transitions, i.e. is $\epsilon$-free, then the transition function can be restricted to $\delta : Q \times \Sigma \rightarrow 2^Q$.

\subsubsection{Deterministic Finite Automata (DFA)} \textit{Deterministic} refers to the uniqueness of the computation. It can be seen as a special kind of NDFA in which for each state and symbol, the result of the transition has exactly one state. Following is the formal definition of a DFA:

\begin{equation*}
    A = (Q,\Sigma,q_0,F,\delta)
\end{equation*}

where, $Q$ is the finite set of states, $\Sigma$ is the finite set of input symbols (the vocabulary), $q_0 \in Q$ is the initial state, $F \in Q$ is the set of accepting (final) states, and $\delta : Q \times \Sigma \rightarrow Q$ is the \textit{transition function}. 

\subsection{Conversion algorithms}
This sections recalls the most prominent algorithms for conversion from a RE to a DFA. 

\subsubsection{Thompson's construction algorithm}
Converts a RE into an equivalent NDFA. The algorithm splits an expression into its constituent sub-expressions and works recursively applying the next rules: 

\begin{itemize}
    \item  $\bf e = \emptyset$ is converted in:
    \begin{figure}[H]
        \begin{center}
        \resizebox{.3\textwidth}{!}{\includegraphics{Images/NdfaREempty.pdf}}
        \end{center} 
    \end{figure}
    
    \item $\bf e = \epsilon$  is converted in:
    \begin{figure}[H]
        \begin{center}
        \resizebox{.1\textwidth}{!}{\includegraphics{Images/NdfaREepsilon.pdf}}
        \end{center} 
    \end{figure}
    
    \item  $\bf e = a$ is converted in:
    \begin{figure}[H]
        \begin{center}
        \resizebox{.3\textwidth}{!}{\includegraphics{Images/NdfaRElit.pdf}}
        \end{center} 
    \end{figure}
    
    \item  $\bf e = p\ .\ q$ is converted in:
    \begin{figure}[H]
        \begin{center}
        \resizebox{.7\textwidth}{!}{\includegraphics{Images/NdfaREthen.pdf}}
        \end{center} 
    \end{figure}
    
    \item  $\bf e = p + q$ is converted in:
    \begin{figure}[H]
        \begin{center}
        \resizebox{.6\textwidth}{!}{\includegraphics{Images/NdfaREor.pdf}}
        \end{center} 
    \end{figure}
    
    \item  $\bf e = p^*$ is converted in:
    \begin{figure}[H]
        \begin{center}
        \resizebox{.6\textwidth}{!}{\includegraphics{Images/NdfaREstar.pdf}}
        \end{center} 
    \end{figure}
\end{itemize}

For instance, the regular expression $\bf e = (\epsilon + a^*b)$ would be converted in the NDFA presented in Figure \ref{fig:ndfaT}

\begin{figure}
    \centering
    \resizebox{.8\textwidth}{!}{\includegraphics{Images/ndfa.png}}
    \caption{NDFA resulting from $\bf e = (\epsilon + a^*b)$ using Thompson's construction }
    \label{fig:ndfaT}
\end{figure}


\subsubsection{Glushkov's construction algorithm}
Is another well-known algorithm to convert a give RE into a NDFA. Is similar to Thompson's construction, once the $\epsilon$-transitions are removed. Following are the construction steps to create a NDFA that accepts the language $L(e)$ accepted by the RE $e$:

\begin{itemize}
    \item \textbf{Step 1} - linearisation of the expression. Each letter of the alphabet appearing in the expression e is renamed, so that each letter occurs at most once in the new expression $e'$. Let $A$ be the old alphabet and let $B$ be the new one;
    \item \textbf{Step 2a} - the following sets are computed:
    \begin{itemize}
        \item $P(e') = \{\ x \in B\quad |\quad xB^* \cap L(e') \neq \emptyset\ \}$  is the set of symbols which occurs as first letter of a word of $L(e')$. Can be defined inductively:
        
        \begin{flalign*}
            P(\emptyset) & = P(\epsilon) = \emptyset \\
            P(a) & = \{a\} \text{, for each letter $a$} \\
            P(s+t) & = P(s) \cup P(t) \\
            P(s\ .\ t) & = P(s)\ \cup \ \Lambda (s) P(t) \\
            P(s^*) & = P(s)
        \end{flalign*}
        
        
        \item $D(e') = \{\ y \in B\quad |\quad B^*y \cap L(e') \neq \emptyset\ \}$ is the set of symbols that can end a word of $L(e')$. Can be defined inductively with the same rules of $P$ except for the product where
        
        \begin{equation*}
            D(s\ .\ t) = D(t)\ \cup \ D(s)\Lambda(t) 
        \end{equation*}
        
        \item $F(e') = \{\ u \in B^2\quad |\quad B^*uB^* \cap L(e') \neq \emptyset\ \}$ is the set f symbol pair that can occur in words of $L(e')$. Can be defined inductively as follows: 
        \begin{flalign*}
            F(\emptyset) & = F(\epsilon) = F(a) = \emptyset \text{, for each letter $a$} \\
            F(s+t) & = F(s) \cup F(t) \\
            F(s\ .\ t) & = F(s)\ \cup\ F(t)\ \cup\ D(s)P(t) \\
            F(s^*) & = F(s)\ \cup\ D(e)P(e)
        \end{flalign*}
        
    \end{itemize}
    
    \item \textbf{Step 2b} - computes the set $\Lambda = \{\epsilon\} \cup L(e')$, in case the empty word belongs to the language, otherwise $\Lambda = \emptyset$. It can be defined inductively for each RE as following:
    \begin{flalign*}
        \Lambda(\emptyset) & = \emptyset \\
        \Lambda(\epsilon) & = \{\epsilon\} \\
        \Lambda(a) & = \emptyset \text{, for each letter $a$} \\
        \Lambda(s+t) & = \Lambda(s) \cup \Lambda(t) \\
        \Lambda(s\ .\ t) & = \Lambda(s)\ .\ \Lambda(t) \\
        \Lambda(s^*) & = \{\epsilon\} 
    \end{flalign*}
    
    
    \item \textbf{Step 3} - computation of the local language, i.e. the set of words which begin with a letter of $P$, end by a letter of $D$ and whose transitions belong to $F$: 
    \begin{equation*}
        L' = (PB^* \cap B^*D)\ \backslash\ B^* (B^2 \backslash F) B^*
    \end{equation*}
    
    \item \textbf{Step 4} - erasing the delinearization, giving to each letter of $B$ the letter of $A$.
\end{itemize}

Given the regular expression $\bf e = (a(ab)^*)^* + (ba)^*$, would produce the NDFA in Figure \ref{fig:ndfaG}, without performing Step 4, to perform this step one only needs to remove the index from each symbol in the nodes.

\begin{figure}
    \centering
    \resizebox{.8\textwidth}{!}{\includegraphics{Images/glushkov.jpg}}
    \caption{NDFA resulting from $\bf e = (a(ab)^*)^* + (ba)^*$ using Glushkov's construction }
    \label{fig:ndfaG}
\end{figure}

\subsubsection{Powerset construction algorithm}
Converts a NDFA into a DFA, which recognize the same formal language. Recall the structure of a NDFA in Section \ref{definitions}, the corresponding DFA has the set of states $Q_D$, where each state corresponds to subsets of $Q_N$. 

The initial state of the DFA is the set $\{q_0\}$ where $q_0$ is the initial state of the NDFA. In case of a NDFA has $\epsilon$-transitions then the initial state of the DFA - $q_{0D}$ - has the initial state of the NDFA - $q_{0N}$ - plus the states reachable from that state through $\epsilon$-transitions - called $\epsilon$-\textit{closure}:
\begin{equation*}
    q_{0D} = q_{0N}\ \cup \ \epsilon\text{-\textit{closure}} \ \{q_{0N}\}
\end{equation*}

Starting from the initial state $q_{0D}$, each new state is computed from another state $S \in Q_D$. It can be defined recursively as:
\begin{itemize}
    \item $q_{0D} \in Q_D$
    \item $qs \in Q_D \Rightarrow \{ d\ |\ (o,y,d) \in \delta_{NDFA} \wedge o \in qs\} \in Q_Dv , \quad \forall y \in \Sigma$
\end{itemize}

The transition function of the DFA maps a state $S$ and the input symbol $y$ to the respective computed state:

\begin{equation*}
    \delta_{DFA} = \cup \{(S,y,D) | D = \{ d\ |\ (o,y,d) \in \delta_{NDFA} \wedge o \in S\}\}, \forall{(y \in \Sigma \wedge S \in Q_D)}
\end{equation*}

Finally, the set of accepting states of the DFA $Z_D$ is the set of states $S \in Q_D$ that contain at least one state that is accepting state in the NDFA:

\begin{equation*}
    Z_D = \{q \in Q_D\ |\ q \cap Z_N \neq \emptyset\}
\end{equation*}

\subsubsection{Kleene's Algorithm}

Kleene's algorithm produces a regular expression from a DFA in the following way.
\begin{enumerate}
    \item It first constructs a graph whose nodes correspond to the states of the automaton and the edges are regular expressions that connect those nodes. 
    \item The initial set of edges contains one edge for each transition in the automaton (labelled with the corresponding literal) plus one edge for each node to itself containing the regular expression $\epsilon$.
    \item It proceeds by applying a variation of the Floyd-Warshall algorithm in order to obtain all paths connecting each of the nodes (summing weights corresponds to concatenation of regular expressions whereas the minimum operation corresponds to alternative ($+$) of regular expressions).
    \item Finally the regular expression equivalent to the original DFA can be obtained as the union ($+$) of the edges which connect the initial state to all final states.
\end{enumerate}
	
    \section{Bidirectional Transformations}
\label{chapter:BX}

Bidirectional Transformations (BX) provide a mechanism for maintaining consistency between two pieces of data. A bidirectional transformation consists of a pair of functions: a \textit{get} function that may discard part of information from a source to produce a view, and a \textit{put} function that accepts a source and a possible modified view, and reflects the changes in the source producing an updated source that is consistent with that view. The pair of functions should be \textit{well-behaved} satisfying the next laws:

\begin{flalign*}
    put\ s\ (get\ s) = s \quad (GetPut) \\
    get\ (put\ s\ v)\ = v \quad (PutGet)
\end{flalign*}

The \textit{GetPut} law requires that if no changes are made on the view \textit{get s} then the \textit{put} function should produce the same unmodified source $s$. The \textit{PutGet} law says that all the changes in the view should be reflected in the source, as so getting from an updated source computed by \textit{put}ting a view $v$ should retrieve the same $v$ ~\cite{bigul,brul}.

The straightforward approach to write a bidirectional transformation is to write two separate functions in any language and show that they satisfy the \textit{well-behavedness} rules. Although this ad hoc solution provides full control over both \textit{get} and \textit{put} transformations, verifying that they are \textit{well-behaved} requires intricate reasoning about their behaviours. Besides, any modification to one of the transformations requires a redefinition of the other transformation as well as a new \textit{well-behavedness} proof ~\cite{bigul,boomerang}. 

A better alternative is to write a single program where both transformations can be described at the same time, using for that a \textit{bidirectional programming language}. Many bidirectional languages have been proposed during the past years and the design challenge for all of them lies in striking a balance between expressiveness and reliability. 

There are two approaches when designing bidirectional programming languages: 

\begin{itemize}
    \item \textit{get-based} - allows the programmer to write the \textit{get} function, deriving a suitable \textit{putback} function. The problem with this approach is that the \textit{get} function may not be injective and so there may exist many possible put functions that can be combined with it to form a valid BX.
    
    \item \textit{putback-based} - allows the programmer to write the \textit{put} from which the only \text{get} function is automatically  derived. The resulting pair of functions must form a \textit{well-behaved} bidirectional transformation.
\end{itemize}

Despite this, as the present work is still exploratory, it focused on studying the possibility of implement a BX between RE and DFA with the first alternative, where the two functions \textit{get} and \textit{put} are written separately. 
	
	\section{Implementation}
	\subsection{Reasoning about BX}
	
	\subsection{Get function from RE to NDFA}
	
	\subsection{Get function from NDFA to DFA}
	
	\subsection{Put back function from DFA to NDFA}
	
	\section{Future work}
	
	
	
	
	
	
	\iffalse
	\section{Regression based approach for link residual time prediction(RLRP)}
	MANET is infrastructure less network form by mobile nodes. These mobile nodes communicate to each other via single hop or multiple hops. Mobility is a key attribute, this causes route failure because the topology of the network changes frequently. So for predicting the future of network topology, link residual time prediction is an important field.
	\par 
	Proposed approach is based on column Mobility Model ~\cite{r15} of mobile nodes it assumes that each node broadcast hello packets to inform its availability in the range. When neighbour node receives hello packet from a sender it uses its signal strength (RSSI Value) to estimate the distance between them.
	------
	monotonically decreasing sequence of RSSI indicates that nodes are moving away from each other and link between them may break down in future. -----.
	\par
	Regression-based approach for link residual time prediction(RLRP) is based on least square polynomial regression. Regression is a better approach than interpolation ~\cite{r16} because interpolation looks for the predicted form of function but in regression looks for a function that minimizes error, this needs a good approximation. 
	
	The following steps are discussed below: 
	
	\begin{enumerate}
		\item Let node $i$ and $j$ are any two neighbour nodes of MANET.
		\item Node $i$  broadcasts hello packets of constant predefined signal strength at periodic intervals.
		\item Neighboring node $j$ receives these packets and  maintains a record of RSSI (received signal strength indication) and packet arrival time in a table. RSSI is a measurement of the signal strength in recieved radio signal. Let $RSSI_{i,j_{1}},RSSI_{i,j_{2}},......RSSI_{i,j_{m}}...\\
		.....RSSI_{i,j_{n}}$ are signal strengths of recieved hello packets by  $j$th node transmitted from  $i$th node recieved at time  $t_{i,j_{1}},t_{i,j_{2}},......t_{i,j_{m}}.....t_{i,j_{n}}$ respectively.
		\item When node $j$ observes a monotonically decreasing pattern of received signal strengths of hello packets as in equation~\ref{eqn:rssi}; In considered case let $m$th onward packets have monotonically  decreasing RSSI valuues then:
		\begin{equation}
		\label{eqn:rssi}
		RSSI_{i,j_{m}}>RSSI_{i,j_{m+1}}>RSSI_{i,j_{m+2}}.....>RSSI_{i,j_{m+n}}
		\end{equation}
		A set ${\Re}_{i,j}$ as equation~\ref{eqn:set} is formulated with above monotonically decreasing hello packet signal strengths and times as elements.
		\begin{equation}
		\label{eqn:set}
		{\Re}_{i,j}= \left\lbrace (RSSI_{i,j_{m}},t_{i,j_{m}}),(RSSI_{i,j_{m+1}},t_{i,j_{m+1}}), \\
		...,(RSSI_{i,j_{m+p}},t_{i,j_{m+p}})...\\
		,(RSSI_{i,j_{m+n}})\right\rbrace 
		\end{equation}
		
		\item Least square polynomial regression based link failure prediction module is invoked.
		\begin{itemize} 
			\item Following is the representation of quadratic model which relates elements of set ~\ref{eqn:set}.
			\begin{equation}% ##-- EQUATION TEMPLATE --##
			\begin{aligned}
			\left[\begin{matrix}RSSI_{i,j_{m}} \\RSSI_{i,j_{m+1}}\\ \vdots\\RSSI_{i,j_{m+p}}\\ \vdots\\RSSI_{i,j_{m+n}}
			\end{matrix} \right] = \left[\begin{matrix} t^{2}_{m} & t_{m} & 1\\t^{2}_{m+1} & t_{m+1} & 1\\   \vdots & \vdots & \vdots\\t^{2}_{m+p} & t_{m+p} & 1 \\  \vdots & \vdots & \vdots  \\t^{2}_{m+n} & t_{m+n} & 1 \end{matrix} \right] 
			*\left[\begin{matrix}  a_{ij} \\ b_{ij} \\ c_{ij}\end{matrix} \right]             
			\end{aligned}\label{eq:six}
			\end{equation}
			\item Equation:~\ref{eq:seven} is simplified representation of equation:~\ref{eq:six}. 
			\begin{equation}% ##-- EQUATION TEMPLATE --##
			\begin{aligned}
			\varUpsilon_{ij}     = \varPhi_{ij} *\varTheta _{ij}
			\end{aligned}\label{eq:seven}
			\end{equation}     
			where 
			\begin{equation}% ##-- EQUATION TEMPLATE --##
			\begin{aligned}
			\varUpsilon_{ij} =\left[\begin{matrix}RSSI_{i,j_{m}} \\RSSI_{i,j_{m+1}}\\ \vdots\\RSSI_{i,j_{m+p}}\\ \vdots\\RSSI_{i,j_{m+n}}
			\end{matrix} \right] , \varPhi_{ij} =\left[\begin{matrix} t^{2}_{m} & t_{m} & 1\\t^{2}_{m+1} & t_{m+1} & 1\\   \vdots & \vdots & \vdots\\t^{2}_{m+p} & t_{m+p} & 1 \\  \vdots & \vdots & \vdots  \\t^{2}_{m+n} & t_{m+n} & 1 \end{matrix} \right] ,       \varTheta _{ij}=\left[\begin{matrix}  a_{ij} \\ b_{ij} \\ c_{ij}\end{matrix} \right] 
			\end{aligned}\label{eq:eight}
			\end{equation} 
			\item Least Square estimate ~\cite{r17} $\hat{\varTheta_{ij}}$ of vector parameter $\varTheta _{ij}$ of equation:~\ref{eq:seven} is following: 
			\begin{equation}% ##-- EQUATION TEMPLATE --##
			\begin{aligned}
			\hat{\varTheta_{ij}}={(\varPhi^{T}\varPhi)}^{-1}\varPhi^{T}\varUpsilon_{ij}
			\end{aligned}\label{eq:eight}
			\end{equation} 
			\item In this way one can estimate values of $\hat{a}_{ij}$,$\hat{b}_{ij}$ and $\hat{c}_{ij}$ and express  signal strength  of $i\Leftrightarrow j$th link $RSSI_{ij}$. Equation:~\ref{eq:nine} represents best fit quadretic polynomial of signal strength with respect to time.
			\begin{equation}% ##-- EQUATION TEMPLATE --##
			\begin{aligned}
			RSSI_{ij} = \hat{a}_{ij} t^{2} + \hat{b}_{ij}t + \hat{c}_{ij}
			\end{aligned}\label{eq:nine}
			\end{equation}        
			\item By antenna characteristics threshold value of signal strength $RSSI_{thresh}$ ~\cite{r12} , min signal strength rquired  to be detected by reciever antenna is a known parameter. Solution of equation:~\ref{eq:ten} will provide estimate of time$\tau_{i,j_{break}}$ when link will break. 
			\begin{equation}% ##-- EQUATION TEMPLATE --##
			\begin{aligned}
			RSSI_{thresh} = \hat{a}_{ij} t^{2} + \hat{b}_{ij}t + \hat{c}_{ij}
			\end{aligned}\label{eq:ten}
			\end{equation}    
			Value of $\tau_{i,j_{break}}$ can be calculated with help of Sridharacharya formulae ~\cite{r13}.
			\begin{equation}% ##-- EQUATION TEMPLATE --##
			\begin{aligned}
			\tau_{i,j_{break}} =\frac{-\hat{b}_{ij} \pm \sqrt{\hat{b}_{ij}^{2}-4\hat{a}_{ij}\hat{c}_{ij}}}{2\hat{a}_{ij}}
			\end{aligned}\label{eq:eleven}
			\end{equation}            
			
			This computed value $\tau_{i,j_{break}}$ is the estimated time when link between $i_{th}$ and $j_{th}$ node will break.  
		\end{itemize}
		\item When Current time= $\tau_{i,j_{break}}- \delta$,  where $\delta$ is very small time; corrective action is taken to prevent packet drop.    
		
		
	\end{enumerate}
	
	
	
	
	
	\section{ Simulation Results and Analysis}
	The proposed approach is based on AODV routing protocol. NS-2.35 ~\cite{r14} tool is used for simulation. NS2.35 is an open source tool which is used for research in networking. 
	
	\subsection{Experimantal setup}
	Column Mobility Model ~\cite{r15} is used for representing mobility of node. In column mobility model, a set of mobile nodes are moving linearly from one location to another location. Velocity of nodes varies from 2 m/s to 10m/s. The simulation area is 500*500, omni-direction antenna model and two-way ray ground radio propagation model are used. In Table~\ref{tab1} detailed parameters for simulation are summarized.
	
	
	\begin{table}
		\caption{Simulation parameters}\label{tab1}
		
		
		\begin{center}
			\begin{tabular}{|c c |} 
				\hline
				Parameters & Values \\ [0.5ex] 
				\hline\hline
				Size & 500 m by 500 m  \\ 
				\hline
				Simulation Time & 500 s \\
				\hline
				Velocity & 2,4,6,8,10 m/s \\
				\hline
				Packet length & 512 bytes \\
				\hline
				Traffic pattern & TCP\\ [1ex] 
				\hline
				MAC protocol & IEEE 802.11  \\ [1ex] 
				\hline
				
				
			\end{tabular}
		\end{center}
		
		
	\end{table}
	
	\subsection{Analysis of results}
	\begin{table}
		\caption{Comparision of estimated vs actual link residual time }\label{tab2}                        
		\begin{center}
			\begin{tabular}{|c c c|} 
				\hline
				Link No & Estimated Residual Time & Actual Residual time  \\   
				\hline\hline
				1 & 252.2639 sec  & 251.9715 sec \\ 
				%                    \hline
				2 & 227.7949 sec  & 226.4358 sec \\
				%                    \hline
				3 & 209.076 sec & 207.9343  sec \\
				%                    \hline
				4 & 204.8224 sec & 204.0284 sec \\
				%                    \hline
				5 & 168.5515 sec & 167.4681 sec \\
				%                    \hline
				6 & 97.8936 sec & 96.5896 sec  \\  
				\hline                    
			\end{tabular}
			
		\end{center}    
	\end{table}    
	Table~\ref{tab2} is representing a comparison of estimated link residual time with actual link residual time for a 5 node manet with node velocity 2m/sec. On observing the data one can conclude that the present algorithm is generating remarkable performance in terms of accurate estimation of link residual time.
	
	\par Root-mean-square error (RMSE) of link residual time  calculated  as following equation:~\ref{eq:e1}is chosen as performance measure:
	\begin{equation}% ##-- EQUATION TEMPLATE --##
	\begin{aligned}
	RMSE_{v}=\sqrt{\dfrac{1}{L}\sum_{k=1}^{L} (\tau_{k,v_{break}}-\tau_{k,v_{actual}})^{2}}
	\end{aligned}\label{eq:e1}
	\end{equation} 
	where    $RMSE_{v}$ is  Root-mean-square error (RMSE) of link residual time at node velocity $v$m/sec, $\tau_{k,v_{break}}$ and  $\tau_{k,v_{actual}}$ are estimated and actual link residual times of any $k_{th}$ link of MANET of total $L$ links with node velocity $v$ respectivey.        
	
	
	\par
	Fig:~\ref{Fig1} shows the comparison of RMSE for the different sets of n points with monotonically decreasing pattern of received signal strengths of hello packet used in the least square polynomial regression wrt velocity of nodes. It can be concluded that increasing node velocity is not directly affecting RMSE value of link residual time; means estimated link residual time is immune to node velocity scaling. 
	\par
	There are three different curves showing RMSE vs node velocity relationship with varying number of hello packets of monotonically decreasing RSSI appearing in the figure:~\ref{Fig1}. It can be concluded that with an increasing value of $n$ there is again in RMSE value of error but since computational overhead is also increasing with $n$. Hence in this algorithm, an optimal value $n=5$ has been chosen.
	
	
	\begin{center}
		\begin{figure}
			
			\includegraphics[width=11cm, height=5.5cm]{r1.jpg}
			\caption{RMSE vs node velocity} \label{Fig1}
			
		\end{figure}
	\end{center}
	\par    
	Fig.~\ref{Fig2} shows a comparison between regression and interpolation based approaches of link residual time estimation. From the figure, it can be concluded that by adapting regression-based approaches one can have an edge over interpolation based approaches. Mathematically, regression-based approaches uses a polynomial pattern within considered points; means there is no guarantee that polynomial will pass through at least one chosen point. On the other hand, interpolation provides curve fitting type solution; means it is guaranteed that a $n_{th}$ degree polynomial will pass through n selected points. In this, although it is guaranteed that there will be zero error at considered n points but for other points in the domain this polynomial will not provide a solution with the optimal error.
	Although in both cases this polynomial is extrapolated as in equation:~\ref{eq:ten} to find a solution for link residual time, yet the regression polynomial is better approximated for this point.
	
	
	\begin{center}
		\begin{figure}
			
			\includegraphics[width=11cm, height=5.5cm]{r2.jpg}
			\caption{Comparision between RMSE of Interpolation vs Regression based residual time estimation} \label{Fig2}
			
		\end{figure}
	\end{center}
	
	%\begin{table}
	%    \caption{Predicted link residual time and actual link failure time when node velocity is 2 m/sec}\label{tab2}
	%    \begin{tabular}{|l|l|l|}
	%        \hline
	%        Heading level &  Example & Font size and style\\
	%        \hline
	%        Title (centered) &  {\Large\bfseries Lecture Notes} & 14 point, bold\\
	%        1st-level heading &  {\large\bfseries 1 Introduction} & 12 point, bold\\
	%        2nd-level heading & {\bfseries 2.1 Printing Area} & 10 point, bold\\
	%        3rd-level heading & {\bfseries Run-in Heading in Bold.} Text follows & 10 point, bold\\
	%        4th-level heading & {\itshape Lowest Level Heading.} Text follows & 10 point, italic\\
	%        \hline
	%    \end{tabular}
	%    \end{table}
	
	
	
	
	
	\section{Conclusion}
	The authors have presented a novel approach for link residual time prediction based upon cross-layer optimization and least square quadratic regression. Although these approaches are very powerful tools of contemporary researchers of networking community yet there are very limited examples of simultaneous use of both for link residual time prediction. Results are very promising and the algorithm is very simple in terms of implementation.
	
	\fi
	
	
	% ---- Bibliography ----
	%
	% BibTeX users should specify bibliography style 'splncs04'.
	% References will then be sorted and formatted in the correct style.
	%
	\bibliographystyle{apalike}
	\bibliography{mybib}
	%
	
\end{document}
